\documentclass[preview]{standalone}

\usepackage[english]{babel}
\usepackage{amsmath}
\usepackage{amssymb}

\begin{document}

\begin{center}
$\lambda_1 = S ( \frac{\beta_1}{\alpha} + \frac{\epsilon_p \beta_4}{\alpha \delta_p} + \frac{\beta_3}{\delta_c} + \frac{\beta1}{\gamma} )$
\end{center}

\end{document}
